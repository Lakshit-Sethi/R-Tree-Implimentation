\documentclass{beamer}
\usepackage[utf8]{inputenc}
\usepackage{graphicx}
\usetheme{Madrid}
\usecolortheme{default}
\usepackage[export]{adjustbox}
\usepackage{amsmath,amssymb}
\usepackage{hyperref}
\usepackage{listings}
\DeclareRobustCommand{\bbone}{\text{\usefont{U}{bbold}{m}{n}1}}

\DeclareMathOperator{\EX}{\mathbb{E}}% expected value
%------------------------------------------------------------
%This block of code defines the information to appear in the
%Title page
\title[R-Tree] %optional
{R-Tree}


\author[] % (optional)
{Dhyey Italiya - 2021A7PS1463P\\Saurabh Bhandari - 2021A7PS2412P\\ Saksham Verma - 2021A7PS2414P\\Lakshit Sethi - 2021A7PS2434P\\Abir Abhyankar - 2021A7PS0523P}

\institute[BITS Pilani] % (optional)
{
  Birla Institute of Technology and Science,Pilani\\
}

%End of title page configuration block
%------------------------------------------------------------


%------------------------------------------------------------
%The next block of commands puts the table of contents at the 
%beginning of each section and highlights the current section

\AtBeginSection[]
{
  \begin{frame}
    \frametitle{Table of Contents}
    \tableofcontents[currentsection]
  \end{frame}
}
%------------------------------------------------------------


\begin{document}

%The next statement creates the title page.
\frame{\titlepage}


%---------------------------------------------------------
%This block of code is for the table of contents after
%the title page
\begin{frame}
\frametitle{Table of Contents}
\tableofcontents
\end{frame}
%---------------------------------------------------------

\section{Introduction}
\begin{frame}
\frametitle{R-Tree Structure}
\begin{itemize}
    \item M - Maximum number of Entries in each Node.
    \item m - Minimum number of Entries in each Node.
    \item MBR - Minimum Bounding Rectangle.It contains two pairs of integers which defines the bounds along X-axis and Y-axis
    \item Node - It contains an array of Entries whose size is bounded by the values of M and m.It also stores a pointer to its Parent Node and parent Entry.
    \item Entry - It contains an MBR and a pointer to its child node.\bigskip
    
    Each node has a boolean isLeaf which checks whether it is a leaf node or not.
\end{itemize}
\end{frame}
\begin{frame}{R-Tree Structure}
    \begin{block}{MBR}
    Minimum Bounding Rectangle.It contains two pairs of integers which defines the bounds along X-axis and Y-axis
         \includegraphics[scale=0.25,center]{MBR_struct.png}
    \end{block}
    \begin{block}{Node}
    It contains an array of Entries whose size is bounded by the values of M and m.It also stores a pointer to its Parent Node and parent Entry.
         \includegraphics[scale=0.25,center]{Node_struct.png}
    \end{block}
\end{frame}

\end{frame}
\begin{frame}{R-Tree Structure}
    \begin{block}{Entry}
    It contains an MBR and a pointer to its child node.
         \includegraphics[scale=0.25,center]{Entry_struct.png}
    \end{block}
    \begin{block}{R-tree}
         \includegraphics[scale=0.25,center]{rTree_struct.png}
    \end{block}
\end{frame}
%------------------------------------------------------------------------
\section{Functions}
\begin{frame}{Functions}
\begin{itemize}
    \item Traversal - We have implemented Preorder Traversal which first prints the node and then its children from left to right.
    \bigskip
    \item Search - It searches for leaf MBRs that overlaps with a given MBR.
    \bigskip
    \item Insert - Insert a given MBR into the tree.
    \bigskip
\end{itemize}
\end{frame}
\begin{frame}{Traversal}
    The function preOrderTraversal uses an auxilliary function named traverse.
    \begin{block}{traverse}
       \includegraphics[scale=0.25,center]{traverse_function.png}
    \end{block}
        
    \begin{block}{PreOrder Traversal}
        \includegraphics[scale=0.25,center]{preOrderTraversal_func.png}
    \end{block}
\end{frame}
%-------------------------
\begin{frame}{Search}
    The function search uses an auxilliary function search utility.
    \begin{block}{search utility}
       \includegraphics[scale=0.25,center]{search_util_func.png}
    \end{block}
\end{frame}
%--------------------------
\begin{frame}{Search}
\begin{block}{search}
       \includegraphics[scale=0.30,center]{search_func.png}
    \end{block}
\end{frame}
%-----------------------------
\begin{frame}{Insertion_{chooseLeaf}}
    \begin{block}{chooseLeaf}
       \includegraphics[scale=0.35,center]{chooseLeaf1_func.png}
    \end{block}
\end{frame}
%-----------------------------
\begin{frame}{Insertion_{chooseLeaf}}
    \begin{block}{chooseLeaf contd.}
       \includegraphics[scale=0.35,center]{chooseLeaf2_func.png}
    \end{block}
\end{frame}
%-----------------------------
\begin{frame}{Insertion_{pickSeeds}}
    \begin{block}{pickSeeds}
       \includegraphics[scale=0.40,center]{pickSeeds1_func.png}
    \end{block}
\end{frame}
%-----------------------------
\begin{frame}{Insertion_{pickSeeds}}
    \begin{block}{pickSeeds contd.}
       \includegraphics[scale=0.35,center]{pickSeeds2_func.png}
    \end{block}
\end{frame}
%-----------------------------
\begin{frame}{Insertion_{pickNext}}
int pickNext(Node *currNode, Entry *group1, Entry *group2, bool *res)
    \begin{block}{pickNext}
       \includegraphics[scale=0.25,center]{pickNext_funcfor.png}
    \end{block}
\end{frame}
%-----------------------------
\begin{frame}{Insertion_{QuadraticSplit}}
void quadraticSplit(Node *currNode, rTree *tree)
    \begin{block}{QuadraticSplit}
       \includegraphics[scale=0.30,center]{QuadSplit1_func.png}
    \end{block}
\end{frame}
%-----------------------------
\begin{frame}{Insertion_{QuadraticSplit}}
Iterating through all the entries of given node with a while loop as:\\
while (currNode --$>$ noOfEntries $>$ 0)\\
Picking the next entry to be added to a group, and the group to which it should be added.
    \begin{block}{QuadraticSplit contd.}
       \includegraphics[scale=0.30,center]{QuadSplit2_func.png}
    \end{block}
\end{frame}
%-----------------------------
\begin{frame}{Insertion_{QuadraticSplit}}
Handling the case where one of the groups has maxChildren entries,we add all the remaining entries to the other group.
    \begin{block}{QuadraticSplit contd.}
       \includegraphics[scale=0.35,center]{QuadSplit3_func.png}
    \end{block}
\end{frame}
%-----------------------------
\begin{frame}{Insertion_{QuadraticSplit}}
If the current node is the root node, create a new root node and add the groups as its children.
    \begin{block}{QuadraticSplit contd.}
       \includegraphics[scale=0.35,center]{QuadSplit4_func.png}
    \end{block}
\end{frame}
%-----------------------------
\begin{frame}{Insertion_{QuadraticSplit}}
Else add the groups as entries to the parent node and remove the current node.
    \begin{block}{QuadraticSplit contd.}
       \includegraphics[scale=0.35,center]{QuadSplit5_func.png}
    \end{block}
\end{frame}
%-----------------------------
\begin{frame}{Insertion_{adjustTree}}
Adjusts the tree after insertion
    \begin{block}{adjustTree}
       \includegraphics[scale=0.40,center]{adjTree1_func.png}
    \end{block}
\end{frame}
%-----------------------------
\begin{frame}{Insertion_{adjustTree}}
Adjust the MBR of the parent node and check if it needs to be split and update the MBR of the parent node.\\
\bigskip
If the parent node needs to be split, split it and adjust the tree again.
    \begin{block}{adjustTree contd.}
       \includegraphics[scale=0.30,center]{adjTree2_func.png}
    \end{block}
\end{frame}
%-----------------------------
\begin{frame}{Insertion}
    \begin{block}{insert}
       \includegraphics[scale=0.35,center]{insert_func.png}
    \end{block}
\end{frame}
%-----------------------------
\end{document}
